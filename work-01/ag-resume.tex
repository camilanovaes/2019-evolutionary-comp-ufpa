\documentclass[12pt]{article}

% Packages:
\usepackage{graphicx}
\usepackage[portuguese]{babel}
\usepackage[utf8]{inputenc}
\usepackage{setspace}
\usepackage{listings}
\usepackage{hyperref}
\usepackage{tocloft}
\usepackage{fancyhdr}
\usepackage{placeins}
\usepackage{subcaption}
\usepackage{subfiles}
\usepackage{outlines}
\usepackage{indentfirst}
\usepackage{amsmath}
\usepackage{enumerate}
\usepackage{subfiles}
\usepackage{color, colortbl, xcolor}
\usepackage{multicol}
%---

% Options:
\setstretch{1} % Espaçamento entre linhas
\usepackage[top=3cm, bottom=2cm, left=1.5cm, right=1.5cm]{geometry}
\PassOptionsToPackage{hyphens}{url}
\title{}
\date{}

% Code customization:
% Default fixed font does not support bold face
\DeclareFixedFont{\ttb}{T1}{txtt}{bx}{n}{10} % for bold
\DeclareFixedFont{\ttm}{T1}{txtt}{m}{n}{10}  % for normal

\lstset{
	basicstyle=\footnotesize,
	columns=fullflexible,
    keywordstyle=\ttb\color{blue},
	stringstyle=\ttm\color{green},
	commentstyle=\color{gray},
	frame=None,
	breaklines=true,
	showstringspaces=false,
	postbreak=\mbox{\textcolor{red}{$\hookrightarrow$}\space},
}

\lstdefinestyle{R} {
	language=R,
	keywordstyle=\ttb\color{blue},      % keyword style
	stringstyle=\ttm\color{purple},     % string literal style
	deletekeywords ={seq,end},
	frame=single,
	numbers=left
}

%---
% Document:
\begin{document}
	% Cabeçalho:
\begin{figure}
		\begin{minipage}{.3\linewidth}
			\centering
			\includegraphics[width=.6\linewidth]{imgs/ufpa.jpg}
		\end{minipage}
		\begin{minipage}{.70\linewidth}
			\flushleft
			\paragraph{}
			\textbf{ }\newline
			\textbf{UNIVERSIDADE FEDERAL DO PARÁ} \newline
			\textbf{INSTITUTO DE TECNOLOGIA} \newline
			\textbf{FACULDADE DE ENGENHARIA DA COMPUTAÇÃO E TELECOMUNICAÇÕES} \newline
			\textbf{TE05205 - Top. Especiais em Engenharia de Computação II} \newline
            \textbf{Prof. Dr. Roberto Celio Limão de Oliveira} \newline
            \textbf{Aluna: Camila Novaes Silva (201606840055)}
		\end{minipage}
\end{figure}
\FloatBarrier
\begin{center}
    {\Large \textbf{Algoritmos Genéticos}}
\end{center}
%%%%%%%%%%%%%

\section{Introdução}


\section{Características e Componentes}
Um algoritimo genetico simples consiste em, primeiramente, inicializar uma população de
possíveis soluções. Posteriormente, cruzar dois indivíduos da população para gerar novos
indivíduos,

\section{Seleção}

\section{Operadores Genéticos}

\section{Exemplo de AG: Container de frutas}

\section{Conclusão}



\end{document}
